



\documentclass[first=dgreen,second=purple,logo=yellowexc]{aaltoslides}
%\documentclass{aaltoslides} % DEFAULT
%\documentclass[first=purple,second=lgreen,logo=redque,normaltitle,nofoot]{aaltoslides} % SOME OPTION EXAMPLES





% input encode
\usepackage[utf8]{inputenc}


%\usepackage[T1]{fontenc}
%\usepackage{lastpage}
%\usepackage{multirow}
%\usepackage{colortbl}
%\usepackage{comment}
%\usepackage{bm}
%\usepackage{natbib}


% Lipsum package generates bullshit
%\usepackage{lipsum}

% Set the document languages
%\usepackage[finnish,swedish,english]{babel}

% nomenclature
%\usepackage[intoc]{nomencl}

% math
\usepackage{amsmath}

% bibliograph
%\usepackage{natbib}

% For algorithms
\usepackage{algorithm}
\usepackage{algorithmic}

% math font
\usepackage{amsfonts}

% theory
%\usepackage{amsthm}

% double bracket
\usepackage{stmaryrd}

% special math symbol
\usepackage{amssymb}

% use enumerate environment
%\usepackage{enumitem}

% use \url \hyperref, make reference clickable
\usepackage{hyperref}

% use lastpage to inde
\usepackage{lastpage}



%-------------------
%
% set
%
%-------------------
\newcommand{\Acal}{\mathcal{A}}
\newcommand{\Bcal}{\mathcal{B}}
\newcommand{\Ccal}{\mathcal{C}}
\newcommand{\Dcal}{\mathcal{D}}
\newcommand{\Ecal}{\mathcal{E}}
\newcommand{\Fcal}{\mathcal{F}}
\newcommand{\Gcal}{\mathcal{G}}
\newcommand{\Hcal}{\mathcal{H}}
\newcommand{\Ical}{\mathcal{I}}
\newcommand{\Jcal}{\mathcal{J}}
\newcommand{\Kcal}{\mathcal{K}}
\newcommand{\Lcal}{\mathcal{L}}
\newcommand{\Mcal}{\mathcal{M}}
\newcommand{\Ncal}{\mathcal{N}}
\newcommand{\Ocal}{\mathcal{O}}
\newcommand{\Pcal}{\mathcal{P}}
\newcommand{\Qcal}{\mathcal{Q}}
\newcommand{\Rcal}{\mathcal{R}}
\newcommand{\Scal}{\mathcal{S}}
\newcommand{\Tcal}{\mathcal{T}}
\newcommand{\Ucal}{\mathcal{U}}
\newcommand{\Vcal}{\mathcal{V}}
\newcommand{\Wcal}{\mathcal{W}}
\newcommand{\Xcal}{\mathcal{X}}
\newcommand{\Ycal}{\mathcal{Y}}
\newcommand{\Zcal}{\mathcal{Z}}

\newcommand{\RR}{\mathbb{R}}
\newcommand{\ZZ}{\mathbb{Z}}

%-------------------
%
% vector
%
%-------------------
\newcommand{\va}{\mathbf {a}}
\newcommand{\vb}{\mathbf {b}}
\newcommand{\vc}{\mathbf {c}}
\newcommand{\vd}{\mathbf {d}}
\newcommand{\ve}{\mathbf {e}}
\newcommand{\vf}{\mathbf {f}}
\newcommand{\vg}{\mathbf {g}}
\newcommand{\vh}{\mathbf {h}}
\newcommand{\vi}{\mathbf {i}}
\newcommand{\vj}{\mathbf {j}}
\newcommand{\vk}{\mathbf {k}}
\newcommand{\vl}{\mathbf {l}}
\newcommand{\vm}{\mathbf {m}}
\newcommand{\vn}{\mathbf {n}}
\newcommand{\vo}{\mathbf {o}}
\newcommand{\vp}{\mathbf {p}}
\newcommand{\vq}{\mathbf {q}}
\newcommand{\vr}{\mathbf {r}}
\newcommand{\vs}{\mathbf {s}}
\newcommand{\vt}{\mathbf {t}}
\newcommand{\vu}{\mathbf {u}}
\newcommand{\vv}{\mathbf {v}}
\newcommand{\vw}{\mathbf {w}}
\newcommand{\vx}{\mathbf {x}}
\newcommand{\vy}{\mathbf {y}}
\newcommand{\vz}{\mathbf {z}}
\newcommand{\vmu}{\mathbf {\mu}}
\newcommand{\valpha}{\mathbf {\alpha}}
\newcommand{\vlambda}{\mathbf {\lambda}}
\newcommand{\vAlpha}{\mathbf {\Alpha}}
\newcommand{\vbeta}{\mathbf {\beta}}
\newcommand{\vBeta}{\mathbf {\Beta}}
\newcommand{\vgamma}{\mathbf {\gamma}}
\newcommand{\vGamma}{\mathbf {\Gamma}}
\newcommand{\vdelta}{\mathbf {\dalta}}
\newcommand{\vDelta}{\mathbf {\Dalta}}
\newcommand{\vone}{\mathbf {1}}
\newcommand{\vzero}{\mathbf {0}}
\newcommand{\vell}{\mathbf {\ell}}
\newcommand{\vxi}{\mathbf{\xi}}
\newcommand{\vphi}{\mathbf{\phi}}
\newcommand{\vPhi}{\mathbf{\Phi}}

%-------------------
%
% math operation
%
%-------------------
\newcommand{\argmax}{\textbf{argmax}}
\newcommand{\argmin}{\textbf{argmin}}
\newcommand{\sign}{\textbf{sign}}
\newcommand{\maximize}{\textbf{max}}
\newcommand{\minimize}{\textbf{min}}
\newcommand{\argkmax}{\textbf{argkmax}}
\newcommand{\argkmin}{\textbf{argkmin}}
\newcommand{\kmaximize}{\textbf{kmax}}
\newcommand{\kminimize}{\textbf{kmin}}
\newcommand{\st}{\textbf{s.t.}}
\newcommand{\set}[1]{\{ #1 \}}
%\newcommand{\ind}[1]{{\llbracket #1 \rrbracket}}
\newcommand{\ind}[1]{\mathbf{1}_{\{#1\}}}
\newcommand{\norm}[1]{\left|\left| #1 \right|\right|}
\newcommand{\ip}[2]{\langle #1, #2 \rangle}
\newcommand{\var}{\textbf{Var}}
\newcommand{\E}{\textbf{E}}
\newcommand{\exponential}[1]{e^{ #1 }}


\newcommand{\Gva}{G_{\va}}
%-------------------
%
% writings
%
%-------------------
\newcommand{\eqdef}{\overset{{\rm \mbox{\tiny def}}}{=}}
\newcommand{\sbf}[1]{\boldsymbol{#1}}
\newcommand{\mbf}[1]{\mathbf{#1}} 
\newcommand{\etal}{{\em et al.}}

\newcommand{\svmstruct}{{\sc ssvm}}
\newcommand{\mmmn}{{\sc m$^3$n}}
\newcommand{\svm}{{\sc svm}}
\newcommand{\mmcrf}{{\sc mmcrf}}
\newcommand{\smo}{{\sc smo}}
\newcommand{\crf}{{\sc crf}}
\newcommand{\nphard}{$\Ncal\Pcal$-hard}
\newcommand{\nphardness}{$\Ncal\Pcal$-hardness}
\newcommand{\iis}{{\sc iis}}
\newcommand{\memm}{{\sc memm}}
\newcommand{\lr}{{\sc lr}}
\newcommand{\svmlight}{{\sc svmlight}}
\newcommand{\libsvm}{{\sc libsvm}}
\newcommand{\svmcascade}{{\sc svmcascade}}
\newcommand{\adaboost}{{\sc adaboost}}
\newcommand{\adaboostmh}{{\sc adaboost.mh}}
\newcommand{\bagging}{{\sc bagging}}
\newcommand{\vrtree}{{\sc vr-tree}}
\newcommand{\deepboosting}{{\sc deepboosting}}
\newcommand{\loo}{{\sc loo}}
\newcommand{\mtl}{{\sc mtl}}
\newcommand{\sdp}{{\sc sdp}}
\newcommand{\iqp}{{\sc iqp}}
\newcommand{\qp}{{\sc qp}}
\newcommand{\daggraph}{{\sc dag}}
\newcommand{\lp}{{\sc lp}}

\newcommand{\hatf}{{\hat{f}}}
\newcommand{\p}{\sc p}
\newcommand{\n}{\sc n}
\newcommand{\pp}{\sc pp}
\newcommand{\pn}{\sc pn}
\newcommand{\nn}{\sc nn}
\newcommand{\maxcut}{{\sc max-cut}}
\newcommand{\greedy}{{\sc greedy}}
\newcommand{\kernelcascade}{{\sc kernel cascade}}
\newcommand{\netrate}{{\sc netrate}}
\newcommand{\netinf}{{\sc netinf}}
\newcommand{\spin}{{\sc spin}}
\newcommand{\vI}{\mathbf{I}}
\newcommand{\tp}{^{\intercal}}
\newcommand{\mve}{{\sc mve}}
\newcommand{\amm}{{\sc amm}}
\newcommand{\mam}{{\sc mam}}
\newcommand{\rta}{{\sc rta}}
\newcommand{\lasso}{{\sc lasso}}
\newcommand{\mle}{{\sc mle}}
\newcommand{\map}{{\sc map}}
\newcommand{\rbf}{{\sc rbf}}
\newcommand{\mlknn}{{\sc ml-knn}}
\newcommand{\knn}{{\sc knn}}
\newcommand{\iblr}{{\sc iblr}}
\newcommand{\cc}{{\sc cc}}
\newcommand{\pcc}{{\sc pcc}}
\newcommand{\ecc}{{\sc ecc}}
\newcommand{\br}{{\sc br}}
\newcommand{\corrlog}{{\sc corrlog}}
\newcommand{\ilgs}{{\sc ilgs}}
\newcommand{\ilrs}{{\sc ilrs}}
\newcommand{\cpp}{{\sc c}}
\newcommand{\matlab}{{\sc matlab}}
\newcommand{\openmp}{{\sc openmp}}
\newcommand{\python}{{\sc python}}
\newcommand{\cvx}{{\sc cvx}}
\newcommand{\lda}{{\sc lda}}
\newcommand{\kkt}{{\sc k.k.t}}
\newcommand{\lbp}{{\sc lbp}}
\newcommand{\anova}{{\sc anova}}

\renewcommand{\algorithmicrequire}{\textbf{Input:}}
\renewcommand{\algorithmicensure}{\textbf{Output:}}



\newcommand{\Upsilonb}{\pmb \Upsilon}
\newcommand{\phib}{\pmb \phi}
\newcommand{\psib}{\pmb \psi}
\newcommand{\varphib}{\pmb \varphi}
\newcommand{\phibh}{\hat\phib}
\newcommand{\psibh}{\hat \psib}
\newcommand{\vYcal}{\pmb \Ycal}
\newcommand{\vXcal}{\pmb \Xcal}
\newcommand{\vFcal}{\pmb \Fcal}
%-------------------
%
% others
%
%-------------------




%\newtheorem{definition}{Definition}
%\newtheorem{theory}{Theory}
%\newtheorem{lemma}{Lemma}

















\title{Newton update in L$_2$-norm random tree approximation}
\author{Hongyu Su}



\institute[ICS]{
Helsinki Institute for Information Technology HIIT\\
Department of Computer Science\\
Aalto University
}

%\aaltofootertext{Random Spanning Tree Approximation}{\today}{\arabic{page}/\pageref{LastPage}\ }
\aaltofootertext{Newton update in \rta}{\today}{\arabic{page}}


\date{ \today} %\date{Version 1.0, \today}

\iffalse
\AtBeginSection[]
{
  \begin{frame}<beamer>{Outline}
    \tableofcontents[currentsection,subsection]
  \end{frame}
}
\fi




%--------------------------------
%
% document
%
%--------------------------------

\begin{document}


\aaltotitleframe
\footnotesize


\begin{frame}{Preliminaries}
	\begin{itemize}\footnotesize
		\item $\Xcal$ is an arbitrary input space, $\vx\in\vXcal$.
		\item $\Ycal$ is an output space of a set of $\ell$-dimensional {\em multilabels}
		\begin{align*}\footnotesize
			\vy=(y_1,\cdots,y_{\ell})\in\vYcal.
		\end{align*}
		\item $y_i$ is a {\em microlabel} and $y_i\in\{1,\cdots,r_i\}, r_i\in\ZZ$.
		\item For example, multilabel binary classification $y_i\in\{-1,+1\}$.
		\item Training examples are sampled from $(\vx,\vy)\in\vXcal\times\vYcal$.
		\item Each example $(\vx,\vy)$ is mapped into a joint feature space $\phib(\vx,\vy)$.
		\item $\vw$ is the weight vector in the joint feature space.
		\item Define a linear score function $F(\vw,\vx,\vy) = \ip{\vw}{\phib(\vx,\vy)}$.
		\item The prediction $\vy_{\vw}(\vx)$ of an input $\vx$ is the multilabel $\vy$ that maximizes the score function 
		\begin{align}\footnotesize
			\vy_{\vw}(\vx) = \underset{\vy\in\vYcal}{\argmax}\,\ip{\vw}{\phib(\vx,\vy)}. \label{inference}
		\end{align}
		\item (\ref{inference}) is called {\em inference} problem which is \nphard\ for most output feature maps.
	\end{itemize}
\end{frame}

\begin{frame}{Markov network}
	\begin{itemize}\footnotesize
		\item We assume that the output feature map $\phib$ is a potential function on a Markov network $G=(E,V)$.
		\item $G$ is a complete graph with $|V| = \ell$ nodes and $|E| = \frac{\ell(\ell-1)}{2}$ undirected edges.
		% \item Joint feature map is $\phib(\vx,\vy) = \varphib(\vx)\otimes\psib(\vy)$.
		\item $\varphib(\vx)$ is the input feature map, e.g., bag-of-words feature of an example $\vx$.
		\item $\psib(\vy)$ is the output feature map which is a collection of edges and labels
		\begin{align*}\footnotesize
			\varphib(\vy) = (u_{e})_{e\in E},u_e\in\{-1,+1\}^2.
		\end{align*}
		\item The joint feature is the Kronecker product of $\varphib(\vx)$ and $\psib(\vy)$
		\begin{align*}\footnotesize
			\phib(\vx,\vy) = (\phib_e(\vx,\vy))_{e\in E}=(\varphib(\vx)\otimes\psib_e(\vy_e))_{e\in E}.
		\end{align*}
		\item The score function is 
		\begin{align*}
			F(\vw,\vx,\vy) = \ip{\vw}{\phib(\vx,\vy)} = \sum_{e\in E}\ip{\vw_e}{\phib_e(\vx,\vy_e)}.
		\end{align*}
	\end{itemize}
\end{frame}

\begin{frame}{Inference in terms of all spanning trees}
	\begin{itemize}
		\item Solving the following inference problem on a complete graph is \nphard
		\begin{align*}
			\vy_{\vw}(\vx) = \underset{\vy\in\vYcal}{\argmax}\,F(\vw,\vx,\vy)  = \underset{\vy\in\vYcal}{\argmax}\,\sum_{e\in E}\ip{\vw_e}{\phib_e(\vx,\vy_e)}. 
		\end{align*}
		\item For a complete graph, there are $\ell^{\ell-2}$ unique spanning trees.
		\item We can write $F(\vw,\vx,\vy)$ as a conic combination of all spanning trees
		\begin{align*}
			F(\vw,\vx,\vy) &= \underset{T\in U(G)}{\E}a_T\ip{\vw_T}{\phib_T(\vx,\vy)}\\
			  &\underset{T\in U(G)}{\E}a_T^2=1,  \underset{T\in U(G)}{\E}a_T<1.
		\end{align*}
		\item $U(G)$ is the uniform distribution over $\ell^{\ell-2}$ spanning trees.
		\item There is a exponential dependency on the number of spanning trees.
	\end{itemize}
\end{frame}

\begin{frame}{A sample of $n$ spanning trees}
	\begin{itemize}\footnotesize
		\item Instead of using all spanning trees, we can just use $n$ spanning trees
		\begin{align*}\footnotesize
			F_{\Tcal}(\vw,\vx,\vy) &= \frac{1}{n}\sum_{i=1}^{n}a_{T_i}\ip{\vw_{T_i}}{\phib_{T_i}(\vx,\vy)}\\
			  &\frac{1}{n}\sum_{i=1}^{n}a_{T_i}^2=1,  \frac{1}{n}\sum_{i=1}^{n}a_{T_i}<1.
		\end{align*}
		\item When
		\begin{align*}\footnotesize
			n\ge\frac{\ell^2}{\epsilon^2}(\frac{1}{16}+\frac{1}{2}\ln\frac{8\sqrt{n}}{\delta}),
		\end{align*}
		with high probability, we have $|F_{\Tcal}(\vw,\vx,\vy)-F(\vw,\vx,\vy)|\le\epsilon$.
		\item A sample of $n\in\Theta(\ell^2/\delta^2)$ random spanning tree is sufficient to estimate the score function.
		\item Margin achieved by $F(\vw,\vx,\vy)$ is also preserved by the sample of $n$ random spanning trees $F_{\Tcal}(\vw,\vx,\vy)$.
	\end{itemize}
\end{frame}


\begin{frame}{Optimization problem}
	\begin{itemize}\footnotesize
		\item The primal optimization problem is defined as
		\begin{align*}\footnotesize
			\underset{\vw_{T_i},\xi_i}{\minimize} & \quad \frac{1}{2}\sum_{i=1}^{n}\norm{\vw_{T_i}}^2 + C\sum_{k=1}^{m}\xi_k\\
			\st & \quad \frac{1}{\sqrt{n}}\sum_{i=1}^{n}{ \langle \vw_{T_i}, \phib_{T_t}(\vx_k,\vy_k) \rangle} - \underset{\vy \neq \vy_k}{\maximize\ } \frac{1}{\sqrt{n}}\sum_{i=1}^{n}{\langle \vw_{T_t}, \phib_{T_i}(\vx_k,\vy) \rangle } \geq 1 -  \xi_k, \\
			& \quad \xi_k\ge0\, , \forall\ k \in \set{1,\dots,m}.
		\end{align*}
		\item The marginalized dual problem is defined as
		\begin{align*}\footnotesize
			\underset{\vmu\in\Mcal}{\maximize} & \quad \sum_{i=1}^{n}\left( \vmu_{T_i}\vell_{T_i} - \frac{1}{2}\vmu_{T_i}K_{T_i}\vmu_{T_i}\right)\\
			\st & \quad \sum_{u_e}\vmu_{T_i,e}(u_e)\le C.
		\end{align*}
	\end{itemize}
\end{frame}

\begin{frame}{Optimization algorithm for a single spanning tree}
	\begin{itemize}\footnotesize
		\item We can solve the optimization problem efficiently for each individual spanning tree. 
		\item The algorithm iterates over all training example until convergence.
		\item For the $k$th iteration:
		\begin{enumerate}\footnotesize
			\item Obtain the solution of the $j$th example in $k$th iteration $\vmu_{T_i}^k(j)$.
			\item Compute the gradient $g_{T_i}^k(j) = \ell_{T_i}(j) - K_{T_i}\vmu_{T_i}^k(j)$.
			\item Compute the update direction 
			\begin{align*}
				\hat{\vmu}_{T_i}^{k+1}(j) = \underset{\vmu\in\Mcal}{\argmax}\,\vmu\tp g_{T_i}^k(j).
			\end{align*}
			\item Compute the difference $\Delta\vmu_{T_i}^{k+1}(j) = \hat{\vmu}_{T_i}^{k+1}(j) - \hat{\vmu}_{T_i}^{k}(j)$.
			\item Perform the update $\vmu_{T_i}^{k+1}(j) = \vmu_{T_i}^k(j) + \tau\Delta\vmu_{T_i}^{k+1}(j)$
		\end{enumerate}
		\item The step size along the update direction $\tau$ is given by the exact line search.
		\begin{align*}
			\frac{\partial \left(f(\vmu_{T_i}^{k+1}(j))-f(\vmu_{T_i}^k(j))\right)}{\partial \tau} = 0, 0\le\tau\le1.
		\end{align*}
	\end{itemize}
\end{frame}


\begin{frame}{$\kappa$-best inference for a collection of $n$ spanning trees}
	\begin{itemize}\footnotesize
		\item The algorithm iterates over all training example until convergence.
		\item For the $k$th iteration:
		\begin{enumerate}\footnotesize
			\item Obtain the solutions of the $j$th example over all trees $(\vmu_{T_i}^k(j)){\color{aaltored}_{i=1}^n}$.
			\item Compute the gradients over all trees $(g_{T_i}^k(j)){\color{aaltored}_{i=1}^n}$.
			\item Compute the update directions
			\begin{align*}
				\hat{\vmu}_{T_i}^{k+1}(j) = \underset{\vmu\in\Mcal}{\argmax}\,\vmu\tp g_{T_i}^k(j),\,{\color{aaltored}\forall i}.
			\end{align*}
			\item Compute the best direction
			\begin{align*}
				\tilde{\vmu}_{T_i}^{k+1}(j) = \underset{\vmu\in(\hat{\vmu}_{T_i}^{k+1}(j)){\color{aaltored}_{i=1}^n}}{\argmax}\,{\color{aaltored}\sum_{i=1}^n}\vmu\tp g_{T_i}^k(j)
			\end{align*}
			\item Compute the difference $\Delta\vmu_{T_i}^{k+1}(j) = \hat{\vmu}_{T_i}^{k+1}(j) - \hat{\vmu}_{T_i}^{k}(j),\,{\color{aaltored}\forall i}$.
			\item Perform the update $\vmu_{T_i}^{k+1}(j) = \vmu_{T_i}^k(j) + \tau\Delta\vmu_{T_i}^{k+1}(j),\,{\color{aaltored}\forall i}.$
		\end{enumerate}
		\item The step size along the update direction $\tau$ is given by the exact line search.
		\begin{align*}\footnotesize
			\frac{\partial \left({\color{aaltored}\sum_{i=1}^n}f(\vmu_{T_i}^{k+1}(j))-{\color{aaltored}\sum_{i=1}^n}f(\vmu_{T_i}^k(j))\right)}{\partial \tau} = 0, 0\le\tau\le1.
		\end{align*}
	\end{itemize}
\end{frame}


\begin{frame}{Update with multiple directions}
	\begin{itemize}\footnotesize
		\item The algorithm iterates over all training example until convergence.
		\item $\vmu:\vmu(j)$, and $g:g(j)$.
		\item For the $k$th iteration:
		\begin{enumerate}\footnotesize
			\item Obtain the solutions of the $j$th example over all trees $(\vmu_{T_i}^k){\color{aaltored}_{i=1}^n}$.
			\item Compute the gradients over all trees $(g_{T_i}^k){\color{aaltored}_{i=1}^n}$.
			\item Compute local update direction from each spanning tree
			\begin{align*}
				{\vmu}_{T_i}^{k,*} = \underset{\vmu\in\Mcal}{\argmax}\,\vmu\tp g_{T_i}^k,\,{\color{aaltored}\forall i}.
			\end{align*}
			\item Project local directions into global directions
			\begin{align*}
				{\vmu}_{T_i}^{G,k,*} \leftarrow {\vmu}_{T_i}^{k,*},\,{\color{aaltored}\forall i}.
			\end{align*}
			\item Define a conic combination of update directions 
			\begin{align*}
				\Delta{\vmu}^{G,k} = \sum_{i=1}^{n}\tau_i \left({\vmu}^{G,k} -{\vmu}_{T_i}^{G,k,*}\right) = \sum_{i=1}^{n}\tau_i \Delta{\vmu}_{T_i}^{G,k,*}
			\end{align*}
			\item Perform the update $\vmu^{G,k+1} = \vmu^{G,k} + \Delta{\vmu}^{G,k+1}.$
			\item Project the global solution on spanning trees $(\vmu_{T_i}^{k+1}){\color{aaltored}_{i=1}^{n}}{\leftarrow}\vmu^{G,k+1}.$
		\end{enumerate}
	\end{itemize}
\end{frame}


\begin{frame}{Newton method to compute $\tau$}
	\begin{itemize}
		\item We want to find $\tau$ that maximize the objective function given the update
		\begin{align*}
			\underset{\tau}{\maximize} & \quad f(\vmu^{G,k} + \Delta{\vmu}^{G,k+1})\\
			\st & \quad 0\le\tau_i\le1.
		\end{align*}
		\item We use Newton method to solve the maximization problem and project the solution into feasible region.
	\end{itemize}
\end{frame}


\begin{frame}{Compute duality gap}
	\begin{itemize}
		\item 
	\end{itemize}
\end{frame}


% \begin{frame}{}
% 	\begin{itemize}
% 	\end{itemize}
% \end{frame}


% \begin{frame}{}
% 	\begin{itemize}
% 	\end{itemize}
% \end{frame}


% \begin{frame}{}
% 	\begin{itemize}
% 	\end{itemize}
% \end{frame}


% \begin{frame}{}
% 	\begin{itemize}
% 	\end{itemize}
% \end{frame}


%
\begin{frame}{Conclusions}
	\begin{itemize}\footnotesize
		\item 
	\end{itemize}
\end{frame}




\iffalse
\begin{frame}[allowframebreaks]{Bibliography}
	%\bibliographystyle{plain}
	\bibliographystyle{apalike}
	\bibliography{example}
\end{frame}
\fi


\end{document}
